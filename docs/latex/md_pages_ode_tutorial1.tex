

\begin{DoxyParagraph}{}
This tutorial shows how to do explicit time integration in pressio.
\end{DoxyParagraph}

\begin{DoxyCode}{0}
\DoxyCodeLine{\textcolor{keywordtype}{int} main(\textcolor{keywordtype}{int} argc, \textcolor{keywordtype}{char} *argv[])}
\DoxyCodeLine{\{}
\DoxyCodeLine{  std::cout << \textcolor{stringliteral}{"{}Running tutorial 2\(\backslash\)n"{}};}
\DoxyCodeLine{}
\DoxyCodeLine{  \textcolor{comment}{/*}}
\DoxyCodeLine{\textcolor{comment}{    We illustrate here how to leverage the package pressio/ode to do}}
\DoxyCodeLine{\textcolor{comment}{    explicit time-\/integration for a system of ODEs with Eigen data types,}}
\DoxyCodeLine{\textcolor{comment}{    which is supported by pressio.}}
\DoxyCodeLine{\textcolor{comment}{}}
\DoxyCodeLine{\textcolor{comment}{    Suppose that you need to use some pressio/ode package for doing explicit time}}
\DoxyCodeLine{\textcolor{comment}{    integration of a system of ODEs which is implemented as in MyApp above.}}
\DoxyCodeLine{\textcolor{comment}{}}
\DoxyCodeLine{\textcolor{comment}{    For the sake of explanation, MyApp at the top of this page is a class}}
\DoxyCodeLine{\textcolor{comment}{    that defines the target system of ODEs and meets}}
\DoxyCodeLine{\textcolor{comment}{    the API needed by pressio to run explicit time integration, i.e.:}}
\DoxyCodeLine{\textcolor{comment}{    (a) it has typedefs that pressio detects for scalar, state and velicity}}
\DoxyCodeLine{\textcolor{comment}{    (b) it has two overleads for the velocity() method (one void and one non-\/void)}}
\DoxyCodeLine{\textcolor{comment}{}}
\DoxyCodeLine{\textcolor{comment}{    As long as the user-\/defined class/app or whatever name you want to call has an}}
\DoxyCodeLine{\textcolor{comment}{    API as the one in MyApp, then it can be readily used with pressio.}}
\DoxyCodeLine{\textcolor{comment}{    Note that if you try to run an explicit time integration with}}
\DoxyCodeLine{\textcolor{comment}{    a user-\/defined system that does not satisfy the target API,}}
\DoxyCodeLine{\textcolor{comment}{    pressio throws a compile-\/time error.}}
\DoxyCodeLine{\textcolor{comment}{}}
\DoxyCodeLine{\textcolor{comment}{    for this tutorial, let us run Forward Euler on the system defined by MyApp.}}
\DoxyCodeLine{\textcolor{comment}{  */}}
\DoxyCodeLine{}
\DoxyCodeLine{  \textcolor{comment}{// *** define some types ***}}
\DoxyCodeLine{  \textcolor{keyword}{using} app\_t       = MyApp;}
\DoxyCodeLine{  \textcolor{keyword}{using} scalar\_t    = \textcolor{keyword}{typename} app\_t::scalar\_type;}
\DoxyCodeLine{  \textcolor{keyword}{using} native\_state\_t  = \textcolor{keyword}{typename} app\_t::state\_type;}
\DoxyCodeLine{  \textcolor{keyword}{using} native\_veloc\_t  = \textcolor{keyword}{typename} app\_t::velocity\_type;}
\DoxyCodeLine{}
\DoxyCodeLine{  \textcolor{comment}{// *** create the app object ***}}
\DoxyCodeLine{  app\_t appObj;}
\DoxyCodeLine{}
\DoxyCodeLine{  \textcolor{comment}{// *** define pressio wrapper types for the state ***}}
\DoxyCodeLine{  \textcolor{comment}{// in this case, pressio behind the scenes detects what type you}}
\DoxyCodeLine{  \textcolor{comment}{// are passing as template argument and since it is not (for now) supported,}}
\DoxyCodeLine{  \textcolor{comment}{// pressio still wraps the object but does not know how to do anythin else.}}
\DoxyCodeLine{  \textcolor{keyword}{using} state\_t = ::pressio::containers::Vector<native\_state\_t>;}
\DoxyCodeLine{}
\DoxyCodeLine{  \textcolor{comment}{// *** create the initial state object ***}}
\DoxyCodeLine{  state\_t y(3);}
\DoxyCodeLine{}
\DoxyCodeLine{  \textcolor{comment}{// *** fill the initial state vector ***}}
\DoxyCodeLine{  \textcolor{comment}{// any pressio wrapper provides the data() method to get a pointer to the wrapped object}}
\DoxyCodeLine{  \textcolor{keyword}{auto} yptr = y.data();}
\DoxyCodeLine{  \textcolor{comment}{// i can now use regular std vector [] operator to fill in}}
\DoxyCodeLine{  (*yptr)[0] = 1.; (*yptr)[1] = 2.; (*yptr)[2] = 3.;}
\DoxyCodeLine{}
\DoxyCodeLine{  \textcolor{comment}{// *** create the pressio stepper ***}}
\DoxyCodeLine{  \textcolor{keyword}{using} ode\_tag = ::pressio::ode::explicitmethods::Euler;}
\DoxyCodeLine{  \textcolor{keyword}{using} stepper\_t = ::pressio::ode::ExplicitStepper<ode\_tag, state\_t, app\_t>;}
\DoxyCodeLine{  stepper\_t stepperObj(y, appObj);}
\DoxyCodeLine{}
\DoxyCodeLine{  \textcolor{comment}{// *** integrated in time ***}}
\DoxyCodeLine{  scalar\_t dt = 0.1;}
\DoxyCodeLine{  ::pressio::ode::advanceNSteps(stepperObj, y, 0.0, dt, 1ul);}
\DoxyCodeLine{}
\DoxyCodeLine{  \textcolor{comment}{// note that for this system and settings, the solution printed should be 2,4,6}}
\DoxyCodeLine{  std::cout << \textcolor{stringliteral}{"{}Computed solution: ["{}}}
\DoxyCodeLine{            << (*yptr)[0] << \textcolor{stringliteral}{"{} "{}} << (*yptr)[1] << \textcolor{stringliteral}{"{} "{}} << (*yptr)[2] << \textcolor{stringliteral}{"{}] "{}}}
\DoxyCodeLine{        << \textcolor{stringliteral}{"{}Expected solution: [2,4,6] "{}}}
\DoxyCodeLine{        << std::endl;}
\DoxyCodeLine{}
\DoxyCodeLine{  \textcolor{keywordflow}{return} 0;}
\DoxyCodeLine{\}}
\end{DoxyCode}


And the system class is\+: 
\begin{DoxyCode}{0}
\DoxyCodeLine{\textcolor{keyword}{struct }MyApp\{}
\DoxyCodeLine{  \textcolor{keyword}{using} scalar\_type   = double;}
\DoxyCodeLine{  \textcolor{keyword}{using} state\_type    = Eigen::VectorXd;}
\DoxyCodeLine{  \textcolor{keyword}{using} velocity\_type = state\_type;}
\DoxyCodeLine{}
\DoxyCodeLine{\textcolor{keyword}{public}:}
\DoxyCodeLine{  \textcolor{keywordtype}{void} velocity(\textcolor{keyword}{const} state\_type \& y,}
\DoxyCodeLine{        scalar\_type t,}
\DoxyCodeLine{        velocity\_type \& R)\textcolor{keyword}{ const}\{}
\DoxyCodeLine{    R(0) = 10. * y(0);}
\DoxyCodeLine{    R(1) = 10. * y(1);}
\DoxyCodeLine{    R(2) = 10. * y(2);}
\DoxyCodeLine{  \};}
\DoxyCodeLine{}
\DoxyCodeLine{  velocity\_type createVelocity()\textcolor{keyword}{ const}\{}
\DoxyCodeLine{    \textcolor{keywordflow}{return} velocity\_type(3);}
\DoxyCodeLine{  \};}
\DoxyCodeLine{\};}
\end{DoxyCode}
 
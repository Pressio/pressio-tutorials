We consider the shallow water equations (SWE) on the spatial domain $\Omega = [-\frac{L}{2},\frac{L}{2}] \times [-\frac{L}{2},\frac{L}{2}]$\+: \[ \begin{split} &\frac{\partial h}{\partial t} + \frac{\partial}{\partial x }( h u) + \frac{\partial}{\partial y }( h v) = 0,\\ &\frac{\partial h u}{\partial t} + \frac{\partial}{\partial x} (h u^2 + \frac{1}{2} \mu_1 h^2) + \frac{\partial}{\partial y }( h u v) = \mu_3 hv,\\ &\frac{\partial h v}{\partial t} + \frac{\partial}{\partial x} (h u v) + \frac{\partial}{\partial y }( h v^2 + \frac{1}{2} \mu_1 h^2) = \mu_3 hu. \end{split} \]

In the above, $h : \Omega \rightarrow \mathbb{R}$ is the height of the water surface, $u : \Omega \rightarrow \mathbb{R}$ is the x-\/velocity, and $v : \Omega \rightarrow \mathbb{R}$ is the y-\/velocity. The system has three parameters\+:
\begin{DoxyItemize}
\item $\mu_1$ is the gravity parameter
\item $\mu_2$ controls the magnitude of the initial pulse
\item $\mu_3$ controls the magnitude of the Coriolis forcing
\end{DoxyItemize}





\begin{DoxyParagraph}{Objective}
To describe, using the SWE as a concrete example, how to use pressio to run a typical end-\/to-\/end ROM workflow.
\end{DoxyParagraph}


\begin{DoxyParagraph}{Organization}
This demo is broken down as follows\+:
\begin{DoxyEnumerate}
\item A step-\/by-\/step walk through of the code to solve the FOM of the SWE\+: \href{./md_pages_swe_fom.html}{\texttt{ here}}
\item A step-\/by-\/step walk through of the code to construct and run a {\itshape standard LSPG}\+: \href{./md_pages_swe_lspg.html}{\texttt{ here}}
\item A step-\/by-\/step walk through of the code to construct and run a {\itshape hyper-\/reduced LSPG}\+: \href{./md_pages_swe_hrlspg.html}{\texttt{ here}}
\item Putting all pieces together\+: step-\/by-\/step walk through to run an end-\/to-\/end ROM workflow\+: \href{./md_pages_swe_endtoend.html}{\texttt{ here}} 
\end{DoxyEnumerate}
\end{DoxyParagraph}



\begin{DoxyParagraph}{}
This page walks through an example code to construct and run a standard LSPG ROM of the SWE. ~\newline
 The full code is available \href{https://github.com/Pressio/pressio-tutorials/blob/swe2d_tutorial/tutorials/swe2d/online_phase/lspg_rom/run_lspg.cc}{\texttt{ here}}.
\end{DoxyParagraph}
First, we include the relevant headers. 
\begin{DoxyCode}{0}
\DoxyCodeLine{\textcolor{preprocessor}{\#include "{}pressio\_rom\_lspg.hpp"{}}}
\DoxyCodeLine{\textcolor{preprocessor}{\#include "{}pressio\_apps.hpp"{}}}
\DoxyCodeLine{\textcolor{preprocessor}{\#include "{}read\_basis.hpp"{}}}
\DoxyCodeLine{\textcolor{preprocessor}{\#include "{}rom\_time\_integration\_observer.hpp"{}}}
\DoxyCodeLine{\textcolor{preprocessor}{\#include "{}CLI11.hpp"{}}}

\end{DoxyCode}


Next, is the main function, where we first process the command line arguments and initialize the pressio logger. 
\begin{DoxyCode}{0}
\DoxyCodeLine{\textcolor{keywordtype}{int} main(\textcolor{keywordtype}{int} argc, \textcolor{keywordtype}{char} *argv[])}
\DoxyCodeLine{\{}
\DoxyCodeLine{  CLI::App app\{\textcolor{stringliteral}{"{}LSPG ROM of 2D Shallow Water Equations"{}}\};}
\DoxyCodeLine{}
\DoxyCodeLine{  \textcolor{keyword}{using} scalar\_t = double;}
\DoxyCodeLine{  \textcolor{keywordtype}{int} N = 64;}
\DoxyCodeLine{  \textcolor{keywordtype}{int} romSizePerDof = 10;}
\DoxyCodeLine{  scalar\_t finalTime = 10.;}
\DoxyCodeLine{  scalar\_t dt = 0.02;}
\DoxyCodeLine{  scalar\_t gravity = 7.5;}
\DoxyCodeLine{  scalar\_t pulse   = 0.125;}
\DoxyCodeLine{  scalar\_t forcing = 0.2;}
\DoxyCodeLine{}
\DoxyCodeLine{  app.add\_option(\textcolor{stringliteral}{"{}-\/k,-\/-\/romSize"{}}, romSizePerDof,}
\DoxyCodeLine{         \textcolor{stringliteral}{"{}Number of modes for each dof to use: default = 10"{}});}
\DoxyCodeLine{}
\DoxyCodeLine{  app.add\_option(\textcolor{stringliteral}{"{}-\/N,-\/-\/numCells"{}}, N,}
\DoxyCodeLine{         \textcolor{stringliteral}{"{}Number of cells along each axis: default = 64"{}});}
\DoxyCodeLine{}
\DoxyCodeLine{  app.add\_option(\textcolor{stringliteral}{"{}-\/T,-\/-\/finalTime"{}}, finalTime,}
\DoxyCodeLine{         \textcolor{stringliteral}{"{}Simulation time: default = 10."{}});}
\DoxyCodeLine{}
\DoxyCodeLine{  app.add\_option(\textcolor{stringliteral}{"{}-\/-\/dt"{}}, dt,}
\DoxyCodeLine{         \textcolor{stringliteral}{"{}Time step size: default = 0.02"{}});}
\DoxyCodeLine{}
\DoxyCodeLine{  app.add\_option(\textcolor{stringliteral}{"{}-\/g,-\/-\/gravity"{}}, gravity,}
\DoxyCodeLine{         \textcolor{stringliteral}{"{}Gravity value(s) to simulate: default = 7.5"{}});}
\DoxyCodeLine{}
\DoxyCodeLine{  app.add\_option(\textcolor{stringliteral}{"{}-\/p,-\/-\/pulse"{}}, pulse,}
\DoxyCodeLine{         \textcolor{stringliteral}{"{}Pulse magnitude, default = 0.125"{}});}
\DoxyCodeLine{}
\DoxyCodeLine{  app.add\_option(\textcolor{stringliteral}{"{}-\/f,-\/-\/forcing"{}}, forcing,}
\DoxyCodeLine{         \textcolor{stringliteral}{"{}Forcing value(s) to simulate: default = 0.2"{}});}
\DoxyCodeLine{}
\DoxyCodeLine{  CLI11\_PARSE(app, argc, argv);}
\DoxyCodeLine{}
\DoxyCodeLine{  pressio::log::initialize(pressio::logto::terminal);}

\end{DoxyCode}


We construct the app object, and read in the bases. For standard LSPG, we need to read in the bases for all mesh points. Here, we assume the basis exists in a local file named {\itshape basis.\+txt}. The following lines of code read in the basis\+: 
\begin{DoxyCode}{0}
\DoxyCodeLine{\textcolor{comment}{// -\/-\/-\/-\/-\/-\/-\/-\/-\/-\/-\/-\/-\/-\/-\/-\/-\/-\/-\/-\/-\/-\/-\/-\/-\/-\/-\/-\/-\/-\/-\/-\/-\/-\/-\/-\/-\/-\/-\/-\/-\/-\/-\/-\/-\/-\/-\/-\/-\/-\/-\/-\/-\/-\/-\/}}
\DoxyCodeLine{\textcolor{comment}{// create FOM object}}
\DoxyCodeLine{\textcolor{comment}{// -\/-\/-\/-\/-\/-\/-\/-\/-\/-\/-\/-\/-\/-\/-\/-\/-\/-\/-\/-\/-\/-\/-\/-\/-\/-\/-\/-\/-\/-\/-\/-\/-\/-\/-\/-\/-\/-\/-\/-\/-\/-\/-\/-\/-\/-\/-\/-\/-\/-\/-\/-\/-\/-\/-\/}}
\DoxyCodeLine{\textcolor{keyword}{using} fom\_t = pressio::apps::swe2d;}
\DoxyCodeLine{\textcolor{keyword}{const} scalar\_t params[3] = \{gravity, pulse, forcing\};}
\DoxyCodeLine{fom\_t appObj(N, params);}
\DoxyCodeLine{}
\DoxyCodeLine{\textcolor{comment}{// -\/-\/-\/-\/-\/-\/-\/-\/-\/-\/-\/-\/-\/-\/-\/-\/-\/-\/-\/-\/-\/-\/-\/-\/-\/-\/-\/-\/-\/-\/-\/-\/-\/-\/-\/-\/-\/-\/-\/-\/-\/-\/-\/-\/-\/-\/-\/-\/-\/-\/-\/-\/-\/-\/-\/}}
\DoxyCodeLine{\textcolor{comment}{// read basis}}
\DoxyCodeLine{\textcolor{comment}{// -\/-\/-\/-\/-\/-\/-\/-\/-\/-\/-\/-\/-\/-\/-\/-\/-\/-\/-\/-\/-\/-\/-\/-\/-\/-\/-\/-\/-\/-\/-\/-\/-\/-\/-\/-\/-\/-\/-\/-\/-\/-\/-\/-\/-\/-\/-\/-\/-\/-\/-\/-\/-\/-\/-\/}}
\DoxyCodeLine{\textcolor{keyword}{using} decoder\_jac\_t   = pressio::containers::MultiVector<Eigen::MatrixXd>;}
\DoxyCodeLine{\textcolor{keyword}{const} \textcolor{keywordtype}{int} romSizeTotal = romSizePerDof*3;}
\DoxyCodeLine{decoder\_jac\_t phi = readBasis(\textcolor{stringliteral}{"{}basis.txt"{}}, romSizeTotal, appObj.numDofs());}
\DoxyCodeLine{\textcolor{keyword}{const} \textcolor{keywordtype}{int} numBasis = phi.numVectors();}
\DoxyCodeLine{\textcolor{keywordflow}{if}( numBasis != romSizeTotal ) \textcolor{keywordflow}{return} 0;}

\end{DoxyCode}


Now that we have the basis, we create a Pressio decoder object. In this instance where we have a linear basis, the decoder, in essence, computes the product $\boldsymbol \Phi \hat{\boldsymbol x} $. We construct a decoder as follows 
\begin{DoxyCode}{0}
\DoxyCodeLine{\textcolor{comment}{// -\/-\/-\/-\/-\/-\/-\/-\/-\/-\/-\/-\/-\/-\/-\/-\/-\/-\/-\/-\/-\/-\/-\/-\/-\/-\/-\/-\/-\/-\/-\/-\/-\/-\/-\/-\/-\/-\/-\/-\/-\/-\/-\/-\/-\/-\/-\/-\/-\/-\/-\/-\/-\/-\/-\/}}
\DoxyCodeLine{\textcolor{comment}{// create decoder obj}}
\DoxyCodeLine{\textcolor{comment}{// -\/-\/-\/-\/-\/-\/-\/-\/-\/-\/-\/-\/-\/-\/-\/-\/-\/-\/-\/-\/-\/-\/-\/-\/-\/-\/-\/-\/-\/-\/-\/-\/-\/-\/-\/-\/-\/-\/-\/-\/-\/-\/-\/-\/-\/-\/-\/-\/-\/-\/-\/-\/-\/-\/-\/}}
\DoxyCodeLine{\textcolor{keyword}{using} native\_state\_t  = \textcolor{keyword}{typename} fom\_t::state\_type;}
\DoxyCodeLine{\textcolor{keyword}{using} fom\_state\_t  = pressio::containers::Vector<native\_state\_t>;}
\DoxyCodeLine{\textcolor{keyword}{using} decoder\_t = pressio::rom::LinearDecoder<decoder\_jac\_t, fom\_state\_t>;}
\DoxyCodeLine{decoder\_t decoderObj(phi);}

\end{DoxyCode}
 Next, we set the reference state for the ROM, $\boldsymbol x^{ref}$. Note that this reference state will be used when we reconstruct the state vector, i.\+e., the state will be reconstructed as \[\hat{ \boldsymbol x} = \boldsymbol \Phi \hat{\boldsymbol x} + \boldsymbol x^{ref}\] In this case we set the reference state to be the initial condition, and set the reference states as 
\begin{DoxyCode}{0}
\DoxyCodeLine{\textcolor{comment}{// -\/-\/-\/-\/-\/-\/-\/-\/-\/-\/-\/-\/-\/-\/-\/-\/-\/-\/-\/-\/-\/-\/-\/-\/-\/-\/-\/-\/-\/-\/-\/-\/-\/-\/-\/-\/-\/-\/-\/-\/-\/-\/-\/-\/-\/-\/-\/-\/-\/-\/-\/-\/-\/-\/-\/}}
\DoxyCodeLine{\textcolor{comment}{// create reference state}}
\DoxyCodeLine{\textcolor{comment}{// -\/-\/-\/-\/-\/-\/-\/-\/-\/-\/-\/-\/-\/-\/-\/-\/-\/-\/-\/-\/-\/-\/-\/-\/-\/-\/-\/-\/-\/-\/-\/-\/-\/-\/-\/-\/-\/-\/-\/-\/-\/-\/-\/-\/-\/-\/-\/-\/-\/-\/-\/-\/-\/-\/-\/}}
\DoxyCodeLine{native\_state\_t yRef(appObj.getGaussianIC(pulse));}

\end{DoxyCode}


We now begin constructing the ROM problem. We start by setting our LSPG state type to be a Pressio vector wrapping an Eigen vector. We note that this is decoupled from the FOM, and we could alternatively use, e.\+g., Kokkos with a host backend. In the following snippet of code, we set this type, initialize the ROM state, and set it to be zero. Note that this last step is due to the fact that we employ the initial condition as our ROM reference state, so the initial conditions for our ROM state are zero. 
\begin{DoxyCode}{0}
\DoxyCodeLine{\textcolor{comment}{// -\/-\/-\/-\/-\/-\/-\/-\/-\/-\/-\/-\/-\/-\/-\/-\/-\/-\/-\/-\/-\/-\/-\/-\/-\/-\/-\/-\/-\/-\/-\/-\/-\/-\/-\/-\/-\/-\/-\/-\/-\/-\/-\/-\/-\/-\/-\/-\/-\/-\/-\/-\/-\/-\/-\/}}
\DoxyCodeLine{\textcolor{comment}{// create ROM problem}}
\DoxyCodeLine{\textcolor{comment}{// -\/-\/-\/-\/-\/-\/-\/-\/-\/-\/-\/-\/-\/-\/-\/-\/-\/-\/-\/-\/-\/-\/-\/-\/-\/-\/-\/-\/-\/-\/-\/-\/-\/-\/-\/-\/-\/-\/-\/-\/-\/-\/-\/-\/-\/-\/-\/-\/-\/-\/-\/-\/-\/-\/-\/}}
\DoxyCodeLine{\textcolor{keyword}{using} lspg\_state\_t = pressio::containers::Vector<Eigen::Matrix<scalar\_t,-\/1,1>>;}
\DoxyCodeLine{\textcolor{comment}{// define ROM state}}
\DoxyCodeLine{lspg\_state\_t yROM(romSizeTotal);}
\DoxyCodeLine{\textcolor{comment}{// initialize to zero (reference state is IC)}}
\DoxyCodeLine{pressio::ops::fill(yROM, 0.0);}

\end{DoxyCode}


We now construct the LSPG problem. To do this, we have to define the time marching scheme, which here we set to be the second-\/order Crank Nicolson method. We define this time stepping scheme and construct the LSPG ROM as follows 
\begin{DoxyCode}{0}
\DoxyCodeLine{\textcolor{comment}{// define LSPG type}}
\DoxyCodeLine{\textcolor{keyword}{using} ode\_tag  = pressio::ode::implicitmethods::CrankNicolson;}
\DoxyCodeLine{\textcolor{keyword}{auto} lspgProblem = pressio::rom::lspg::createDefaultProblemUnsteady}
\DoxyCodeLine{  <ode\_tag>(appObj, decoderObj, yROM, yRef);}

\end{DoxyCode}


Now, we need to define the linear and nonlinear solvers use in our LSPG problem. Here, we employ a least-\/squares conjugate gradient solver to use within the Gauss--Newton solver for our LSPG ROM, where we additionally set the convergence tolerance and maximum number of iterations 
\begin{DoxyCode}{0}
\DoxyCodeLine{\textcolor{comment}{// linear solver}}
\DoxyCodeLine{\textcolor{keyword}{using} eig\_dyn\_mat = Eigen::Matrix<scalar\_t, -\/1, -\/1>;}
\DoxyCodeLine{\textcolor{keyword}{using} hessian\_t   = pressio::containers::DenseMatrix<eig\_dyn\_mat>;}
\DoxyCodeLine{\textcolor{keyword}{using} solver\_tag      = pressio::solvers::linear::iterative::LSCG;}
\DoxyCodeLine{\textcolor{keyword}{using} linear\_solver\_t = pressio::solvers::linear::Solver<solver\_tag, hessian\_t>;}
\DoxyCodeLine{linear\_solver\_t linSolverObj;}
\DoxyCodeLine{}
\DoxyCodeLine{\textcolor{comment}{// GaussNewton solver with normal equations}}
\DoxyCodeLine{\textcolor{keyword}{auto} solver = pressio::rom::lspg::createGaussNewtonSolver(lspgProblem, yROM, linSolverObj);}
\DoxyCodeLine{solver.setTolerance(1e-\/10);}
\DoxyCodeLine{solver.setMaxIterations(10);}

\end{DoxyCode}


Next, we construct an observer.


\begin{DoxyCode}{0}
\DoxyCodeLine{\textcolor{comment}{// create observer (see rom\_time\_integration\_observer.hpp in parent dir)}}
\DoxyCodeLine{observer<lspg\_state\_t,native\_state\_t> Obs(yRef);}

\end{DoxyCode}
 Observers act like hooks, and are called at the end of each time step/minimization problem. The observer class can be found \mbox{[}here\mbox{]}(todo). When initialized, our observer writes the reference state to file in {\ttfamily state\+\_\+ref.\+bin}. When this observer is called after each time step, it will then write the ROM solution to a {\ttfamily solution.\+bin} file.

Next we define information for our time marching, e.\+g. initial time, time step, end time, and total number of time steps, and solve the problem. The code advances the ROM in time for {\itshape Nsteps} time steps. Note that, as LSPG consists of solving a series of residual minimization problems, so we call this {\ttfamily solve\+NSequential\+Minimizations}\+: 
\begin{DoxyCode}{0}
\DoxyCodeLine{  \textcolor{comment}{// solve}}
\DoxyCodeLine{  \textcolor{keyword}{const} \textcolor{keyword}{auto} Nsteps = \textcolor{keyword}{static\_cast<}::pressio::ode::types::step\_t\textcolor{keyword}{>}(finalTime/dt);}
\DoxyCodeLine{  \textcolor{keyword}{auto} startTime = std::chrono::high\_resolution\_clock::now();}
\DoxyCodeLine{  pressio::rom::lspg::solveNSequentialMinimizations(lspgProblem, yROM, 0.0, dt, Nsteps, Obs, solver);}
\DoxyCodeLine{  \textcolor{keyword}{auto} finishTime = std::chrono::high\_resolution\_clock::now();}
\DoxyCodeLine{  \textcolor{keyword}{const} std::chrono::duration<double> elapsed2 = finishTime -\/ startTime;}
\DoxyCodeLine{  std::cout << \textcolor{stringliteral}{"{}Walltime (single ROM run) = "{}} << elapsed2.count() << \textcolor{charliteral}{'\(\backslash\)n'};}
\DoxyCodeLine{}
\DoxyCodeLine{  \textcolor{keywordflow}{return} 0;}
\DoxyCodeLine{\}}

\end{DoxyCode}


This completes our description of writing the LSPG coupler. \#\+Click \href{./md_pages_tutorials_tutorial3.html}{\texttt{ here}} to return to the SWE tutorial. 